\documentclass{beamer}

%%%%%%%%%%%%%%%%%%%%%%%%%%%%%%%%%%
% PAKCAGES
%%%%%%%%%%%%%%%%%%%%%%%%%%%%%%%%%%

\usepackage{etoolbox}
\usepackage{xparse}
\usepackage{graphicx}
\usepackage{subcaption}
\usepackage{moresize}
\usepackage{anyfontsize}

\usepackage{calculator}



\makeatletter

% add input source files
% set the input path.
% EXAMPLE \setInputPath{{src/images}{src/tikzs}}
\NewDocumentCommand{\setInputPath}{m}{%
    \def\input@path{#1}%
}

\NewDocumentCommand{\setFontSize}{m o m}{%
    \IfNoValueF{#2}{%
        \fontsize{#1}{#2}\selectfont#3%
    }{%
        % see https://texblog.org/2012/08/29/changing-the-font-size-in-latex/
        \MULTIPLY{#1}{1.2}{\setFont@baseline}%
        \fontsize{#1}{\setFont@baseline}\selectfont#3%
    }%
}

\makeatother
\NewDocumentCommand{\CPD}{}{%
    \texttt{CPD}%
}

\NewDocumentCommand{\ALT}{}{%
    \texttt{ALT}%
}

\NewDocumentCommand{\AWA}{}{%
    \texttt{AWA$^*$}%
}

\NewDocumentCommand{\WA}{}{%
    \texttt{WA$^*$}%
}

\NewDocumentCommand{\A}{}{%
    \texttt{A$^*$}%
}

\NewDocumentCommand{\CPDSearch}{}{%
    \texttt{CPD-Search}%
}

\NewDocumentCommand{\anytimeCPDSearch}{}{%
    \texttt{Anytime CPD-Search}%
}


\NewDocumentCommand{\CPDPathName}{}{%
    \texttt{CPD}-Path%
}

\NewDocumentCommand{\CPDPathsName}{}{%
    \texttt{CPD}-Paths%
}

\NewDocumentCommand{\CPDPath}{m m}{%
    \texttt{CPD-Path}[#1, #2]%
}

\NewDocumentCommand{\CPDPathCostOriginal}{m m}{%
    \ifmmode{h_{CPD}[#1]}\else{$h_{CPD}[#1]$}\fi%
}

\NewDocumentCommand{\CPDPathCostNew}{m m}{%
    \ifmmode{h'_{CPD}[#1]}\else{$h'_{CPD}[#1]$}\fi%
}

\NewDocumentCommand{\pathOnGraph}{m m}{%
    path[#1, #2]%
}

\usetheme{Frankfurt}

\title{Path Planning with CPD Heuristics}
%\subtitle{}

\author{Massimo Bono$^1$, Alfonso E. Gerevini$^1$, Daniel D. Harabor$^2$ and Peter J.Stuckey$^2$}
\institute{%
    $^1$\setFontSize{7.8}{Dipartimento di Ingegneria dell'Informazione, Università degli Studi di Brescia, Italy}%
    \\%
    $^2$\setFontSize{7.8}{Faculty of Information Technology, Monash University, Melbourne, Australia}%
    \\%
    \{mbono, alfonso.gerevini\}@unibs.it, \{daniel.harabor, peter.stuckey\}@monash.edu%
}
%\date{\today}
\date{August 15, 2019}

\setInputPath{%
    {src/texs}%
    {src/images}%
    {src/tikzs}%
    {src/bibs}%
}

\begin{document}

% High level
% SLIDE 1: titolo
% SLIDE 2: 
%    - path finding è importante (videogiochi, routing); moderne soluzioni usano auxiliary data per calcolare velocemente;
%    - una variante del problema è quella in cui i costi degli archi non sono fissi, ma cambiano. 
%    - 
path planning is really important, single agents normally navigate through a static map, but what happens in dynamic context (original map is perturbated via non decreasing perturbation?)
% SLIDE 3: esempio per motivare il lavoro: siamo su una mappa stradale e stiamo seguendo un percorso ottimo. Ad un certo punto
% sul nostro percorso ottimo scopriamo che  c'è un ingorgo che ci rallenterebbe. Abbiamo quindi la scelta di:
% ricalcolare il nostro percorso ottimo da zero sulla nuova mappa (mappa originale + perturbazioni) oppure possiamo
% cercare di calcolare il percorso ottimo sfruttando le informazioni precedenti (mappa original senza perturbazioni)
% SLIDE 4: table of contents
%   background: CPD, ALT, AWA*
% SLIDE 5: CPD (citazione)
%   faccio vedere come lo costruisco, come lo interrogo nelle query di path finding, complessità;
% SLIDE 6: ALT (citazione)
%   descrivo l'euristica. Dico come abbiamo fatto a posizionare i landmark;
% SLIDE 7: AWA* (citazione)
%   descrivo l'idea. 



\begin{frame}[plain]
    \titlepage
    %TODO add brescia and monash logos
    \begin{minipage}{0.5\textwidth}
        \begin{figure}
            \centering
            \includegraphics[width=0.8\textwidth]{src/images/monash}
        \end{figure}
    \end{minipage}\hfill%
    \begin{minipage}{0.5\textwidth}
        \begin{figure}
            \centering
            \includegraphics[width=0.8\textwidth]{src/images/unibs}
        \end{figure}
    \end{minipage}%
\end{frame}
\input{src/texs/02-Introduction}


\end{document}
